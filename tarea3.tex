% ====== TAREA 3 DE PROBABILIDAD ======

\documentclass[12pt,a4paper]{report}
\usepackage[utf8x]{inputenc}
\usepackage{amsmath}
\usepackage{amsfonts}
\usepackage{amssymb}
\usepackage{graphicx}
\usepackage{enumitem}

\newcommand*{\Comb}[2]{{}^{#1}C_{#2}}

\begin{document}
\begin{titlepage}
	\centering
	{\scshape\LARGE Universidad Autónoma de México \par}
	\vspace{1cm}
	{\scshape\Large Probabilidad I\par}
	\vspace{1.5cm}
	{\huge\bfseries Tarea III\par}
	\vspace{.5cm}
	{\Large\itshape Sandra Del Mar Soto Corderi \par}
	\vspace{.5cm}
	{\Large\itshape Edgar Quiroz Castañeda \par}
    \vspace{.5cm}
	{\Large\itshape Raúl Llamosas Alvarado \par}
	 \vspace{.5cm}
	{\Large\itshape Alan Ernesto Arteaga Vázquez \par}
	\vfill
	 \includegraphics[width=0.5\textwidth]{escudo.png}
	\vfill

% Bottom of the page
	{\large Martes 4 de septiembre del 2018 \par}
\end{titlepage}

\pagebreak
\setlength{\voffset}{-0.75in}
\setlength{\headsep}{5pt}

\begin{enumerate}
   % Ejercicio 1
   \item {
  Una urna contiene cinco bolas numeradas del 1 al 5 de las cuales las primeras tres son negras y las últimas dos son doradas. Se extrae una muestra con reemplazo de tamaño dos. Sea $B_{1}$ el evento de que la primera bola extraída sea nera y sea $B_{2}$ el evento de que la segunda bola sea negra.
\begin{enumerate}[label=\alph*) ]
	%a
	\item{Describir el espacio muestral para este experimento y exhibir los eventos $B_{1},B_{2}$ y $B_{1}\cap B_{2}$ \\

	}

	%b
	\item{Encontrar $P(B_{1}),P(B_{2}),P(B_{1}\cap B_{2})$\\

	}

	%c
	\item{Repetir los incisos (a),(b) para un muestreo sin reemplazo. \\

	}



\end{enumerate}

	}

	% Ejercicio 2
   \item {
    Diga si las siguientes afirmaciones son verdaderas o falsas. Demuestre o de un contraejemplo:\\

    \begin{enumerate}[label=\alph*) ]
	%a
	\item{ $P(A|B) \leq P(A)$\\

	}

	%b
	\item{Si $P(A|B) \geq P(A) \Rightarrow P(B|A)\geq P(B)$\\

	}

	%c
	\item{Si $P(A)=0 \Rightarrow A= \emptyset$ \\

	}

	\item{Si $P(A)=P(B^c) \Rightarrow A=B^c$}\\

	\item{$P(A|B) + P(A|B^c)=1$}\\

	\item{Si A y B son mutuamente excluyentes entonces $P(A|A \cup B)=\frac{P(A)}{P(A)+P(B)}$}



\end{enumerate}

	}


	% Ejercicio 3
   \item {

	\begin{enumerate}[label=\alph*) ]
	% a)
   \item {
	Se dice que un evento B carga informacion negativa acerca del evento A si $P(A|B) \leq P(A)$, se denota  $B \searrow A$. Demuestre o de un contraejemplo a las siguientes afirmaciones:
		\begin{enumerate}
		\item{
			Si $B \searrow A \Rightarrow A \searrow B$ \\
			Tenemos que
			\[B \searrow A \implies P(A|B) = \frac{P(A \cap B)}{P(B)} \leq P(A) \implies \frac{P(A \cap B)}{P(A)} \leq P(B)\]
			Y como $\frac{P(A \cap B)}{P(A)} = P(B|A)$, entonces
			\[\frac{P(A \cap B)}{P(A)} = P(B|A) \leq P(B) \implies A \searrow B\]
		}\\
		Para ambos incisos restantes, sea $\Omega = \{1,2, 3, 4, 5, 6\}$ el espacio muestral
		\item{
			Si $B \searrow A \ \wedge A \searrow C \ \Rightarrow B \searrow C$\\
			Con $B = \{4, 5, 6\}$, $A = \{1, 3, 5\}$ y $C = \{2, 4, 6\}$ se tiene que
			\[P(A) = \frac{1}{2}, P(B) = \frac{1}{2}, P(C) = \frac{1}{2}\]
			Y de esto se obtiene que
			\begin{align*}
				&P(A|B) = \frac{P(A \cap B)}{P(B)} = \frac{\frac{1}{6}}{\frac{1}{2}}
				= \frac{1}{3} \leq \frac{1}{2} = P(A) \implies B \searrow A\\
				&P(C|A) = \frac{P(C \cap A)}{P(A)} = \frac{0}{\frac{1}{2}}
				= 0 \leq \frac{1}{2} = P(C) \implies A \searrow C\\
				&P(C|B) = \frac{P(C \cap B)}{P(B)} = \frac{\frac{1}{3}}{\frac{1}{2}}
				= \frac{2}{3} \geq \frac{1}{2} = P(C) \implies B \nearrow C
			\end{align*}
			Por lo tanto, la proposición es falsa
			}\\
		\item{
			Si $B \searrow A \ \wedge C \searrow A \ \Rightarrow B\cap C \searrow A$\\
			Con $A = \{2, 5, 6\}$, $B = \{1, 2, 3\}$ Y $C = \{2, 4\}$ se tiene que
			\[P(A) = \frac{1}{2}, P(B) = \frac{1}{2}, P(C) = \frac{1}{3}\]
			Y de aquí
			\begin{align*}
				&P(A|B) = \frac{P(A \cap B)}{P(B)} = \frac{\frac{1}{6}}{\frac{1}{2}}
				= \frac{1}{3} \leq \frac{1}{2} = P(A) \implies B \searrow A\\
				&P(A|C) = \frac{P(A \cap C)}{P(C)} = \frac{\frac{1}{6}}{\frac{1}{3}}
				= \frac{1}{2}\leq \frac{1}{2} = P(A) \implies C \searrow A\\
				&P(A|B \cap C) = \frac{P(A \cap C \cap B)}{P(B \cap C)} = \frac{\frac{1}{6}}{\frac{1}{6}}
				= 1 \geq \frac{1}{2} = P(A) \implies B \cap C \nearrow A
			\end{align*}
			Por lo tanto la proposición es falsa
		}
		\end{enumerate}
   }

   % b)
   \item {
 Se dice que un evento B carga informacion positiva acerca del evento A si
 $P(A|B) \geq P(A)$ se denota como $B \nearrow A$. Demuestre o de un contraejemplo a las siguientes afirmaciones:
 \begin{enumerate}
 \item{
 	Si $B\nearrow A \ \Rightarrow A\nearrow B$\\
	Análogamente,
	\[B \nearrow A \implies P(A|B) = \frac{P(A \cap B)}{P(B)} \geq P(A) \implies \frac{P(A \cap B)}{P(A)} \geq P(B)\]
	Y como $\frac{P(A \cap B)}{P(A)} = P(B|A)$, entonces
	\[\frac{P(A \cap B)}{P(A)} = P(B|A) \geq P(B) \implies A \nearrow B\]
	}\\

	Para ambos incisos restantes, sea $\Omega = \{1,2, 3, 4, 5, 6\}$ el espacio muestral
 \item{
 		Si $B \nearrow A \ \wedge A \nearrow C \ \Rightarrow B \nearrow C$\\
		Con $A = \{2, 4, 6\}$, $B = \{2, 3, 4, 5, 6\}$ y $C = \{1, 2, 3, 6\}$ tenemos que
		\begin{align*}
			&P(A|B) = \frac{P(A \cap B)}{P(B)} = \frac{\frac{1}{2}}{\frac{5}{6}}
			= \frac{3}{5} \geq \frac{1}{2} = P(A) \implies B \nearrow A \\
			&P(C|A) = \frac{P(A \cap C)}{P(A)} = \frac{\frac{1}{3}}{\frac{1}{2}}
			= \frac{2}{3} \geq \frac{2}{3} = P(C) \implies A \nearrow C \\
			&P(C|B) = \frac{P(C \cap B)}{P(B)} = \frac{\frac{1}{2}}{\frac{5}{6}}
			= \frac{3}{5} \leq \frac{2}{3} = P(C) \implies B \searrow C
		\end{align*}
		Por lo tanto la proposición es falsa\\
	}
 \item{
 		Si $B \nearrow A \ \wedge C \nearrow A \ \Rightarrow B\cap C \nearrow A$\\
		Con $A = \{1, 2, 3, 6\}$, $B = \{2, 4, 6\}$ Y $C = \{1, 2, 3, 4\}$ pasa que
		\begin{align*}
			&P(A|B) = \frac{P(A \cap B)}{P(B)} = \frac{\frac{1}{3}}{\frac{1}{2}}
			= \frac{2}{3} \geq \frac{2}{3} = P(A) \implies B \nearrow A\\
			&P(A|C) = \frac{P(A \cap C)}{P(C)} = \frac{\frac{1}{2}}{\frac{2}{3}}
			= \frac{3}{4} \geq \frac{2}{3} = P(A) \implies C \nearrow A\\
			&P(A|B \cap C) = \frac{P(A \cap B \cap C)}{P(B \cap C)} = \frac{\frac{1}{6}}{\frac{1}{3}}
			= \frac{1}{2} \leq \frac{2}{3} = P(A) \implies B \cap C \searrow A
		\end{align*}
		Por lo tanto, la proposición es false
	}
 \end{enumerate}

   }




	\end{enumerate}

	}

	% Ejercicio 4
   \item {
  	Se tienen 100 urnas de tres diferentes tipos. El primer tipo contiene 8 bolas blancas y 2 negras; el segundo tipo 4 blancas y 6 negras; el tercero 1 blanca y 9 negras. Se selecciona una urna al azar y se extrae de ella una bola que resulta blanca. Se devuelve la bola a la urna y se repite el proceso, siendo ahora la bola extraída negra. Si sabe que la probabilidad de que siendo la bola blanca proceda el primer tipo de urna es $\frac{16}{36	}$ y la probabilidad de que siendo la bola negra proceda del segundo tipo de urna es $  \frac{30}{61}$. Calcular el número de urnas de cada tipo.
	}

	% Ejercicio 5
   \item {
   Demuestre que si A,B,C son independientes entonces A y $B\cup C$ son independientes.

    }

	% Ejercicio 6
   \item {
   Se sabe que $P(A)= \frac{1}{2}$ y $P(A\cup B)=\frac{7}{10}$. Alex supone que A y B son independientes y calcula $P(B)$ basándose en dicha suposición. Augusto supone que A y B son mutuamente excluyentes y calcula $P(B)$ basándose en dicha suposición. Encontrar la diferencia absoluta entre los dos resultados.
	}

	% Ejercicio 7
   \item {
   Considérese $(\Omega,F, P)$ un esacio de probabilidad. Demuestre que si $\lbrace A_{i} \rbrace _{i=1}^{n} \subset F$ son eventos independientes entonces:\\
   \begin{center}
   $P(\bigcup\limits_{i=1}^{n} A_{i})=1-\prod\limits_{i=1}^{n} (1-P(A_{i}))$
   \end{center}
	}

	% Ejercicio 8
   \item {
    Sea $\Omega = \lbrace 1,2,3,4 \rbrace$ y supongase que $P(\lbrace i \rbrace)=.25$ para cualquier $i\in \rbrace 1,2,3,4 \rbrace$. Demuestre que las parejas de eventos A y B, A y C, B y C son independientes. Donde:\\
    \begin{center}
    $A=\lbrace 1,2,3,4 \rbrace \ \ B=\lbrace 2,3 \rbrace \ \ C= \lbrace 2,4 \rbrace $
\end{center}
	}

	% Ejercicio 9
   \item {
  Exhiba un ejemplo de un espacio de probabilidad de tal manera que $A,B,C$ satisfagan que: \\
  \begin{center}
  $P(A\cap B \cap C) = P(A)P(B)P(C)$
  \end{center}
  pero que no sean independientes.
	}

	% Ejercicio 10
   \item {
    Se presentan tres problemas clasicos de probabilidad.

    \begin{enumerate}[label=\alph*) ]
    \item{Se lanzan tres dados balanceados. Comparar la probabilidad de que la suma de las caras sea 8 con la probabilidad de que la suma de las caras sea 9.
    }\\

    \item{Comparar la probabilidad de que al menos un 5 aparezca en cuatro lanzamientos de un dado balanceado con la probabilidad de que aparezca un doble 5 en 24 lanzamientos de dos dados balanceados.} \\
    \item{Comparar la probabilidad de que al menos un 3 aparezca cuando seis dados son lanzados con la probabilidad de que al menos dos 3s aparezcan cuando doce dados son lanzados.}
    \end{enumerate}
	}


	% Ejercicio 11
   \item {
   De un ejemplo de un espacio de probabilidad tal que tres eventos $A_{1},A_{2},A_{3}$ satisfagan que $P(A_{i}\cap A_{j})=P(A_{i})P(A_{j})$ para $i\neq j$ pero que no sean independientes.\\
	}

% Ejercicio 12
   \item {En el salón de un cierto grupo de Probabilidad I hay 4 hombres y 6 mujeres que estudian la licenciatura en Actuaría. También hay 6 hombres que estudian la licenciatura en Matemáticas. ¿Cuántas mujeres de la licenciatura en Matemáticas deben estar presentes en la clase si el sexo y la carrera se consideran independientes cuando se selecciona un estudiante al azar?
	}

	  %Ejercicio 13
  \item{
 Resuelva los siguientes ejercicios:\\
 \begin{enumerate}[label=\alph*) ]
 \item{Se sabe que $P(A)= \frac{2}{5},P(A\cup B)=\frac{3}{5},P(B|A)=\frac{1}{4},P(C|B)=\frac{1}{3}$ y $P(C|A\cap B)=\frac{1}{2}$. Encontrar $P(A|B\cap C)$}\\
 \item{Se sabe que $P(A|B\cap C)=0.6, P(B|A\cap C)=0.3$ y $P(C|A\cap B)=0.9$ encontrar: \\
 \begin{center}
 $P(A\cap B\cap C| (A\cap B) \cup (A\cap C) \cup (B\cap C)$
 \end{center} }
 \end{enumerate}
  }


		  %Ejercicio 14
  \item{
 Una caja contiene bolas numeradas del 1 al n de modo que al escoger una bola al azar la probabilidad de obtener un numero es proporcional a su magnitud. Se selecciona una bola al azar ¿Cuál es la probabilidad de que esté marcada con el número 1\\
 \begin{enumerate}[label= \alph*) ]
 \item{Sin tener informacion adiconal?}\\
 \item{Sabiendo que la bola seleccionada está marcada con algún número entre 1 y m$\leq m \leq n$?}
 \end{enumerate}
  }

  		  %Ejercicio 15
  \item{
Un alumno está buscando una tarjeta para regalarle a su novia, dicho alumno tiene acceso a tres tiendas de regalos, cada una de las tiendas opera en forma independiente de las otras. Cada tienda tiene $\frac{1}{2}$ de probabilidad de tener el regalo buscado y si la tienda tiene el regalo la probabilidad de que el regalo se encuentre agotado es de $\frac{1}{2}$. Encuentre la probabilidad de que el alumno consiga el regalo en alguna de las tres tiendas.
  }






\end{enumerate}
\end{document}
